

\section{Testing}
\label{sec:testing}
\index{Testing}
Testing should be an integral part of developing code.

\subsection{Unit testing}
\index{Testing!unit testing}
\index{Xenon!sourceSet!test@\texttt{test}}
% TODO Unit testing section

% in the package explorer, expand src/test/java and open nl.esciencecenter.xenon.XenonFactoryTest.
% eclipse opens JUnit pane with test results.

Running unit tests
\begin{lstlisting}[style=basic,style=bash,escapeinside={(*@}{@*)}]
cd (*@\mytilde@*)/Xenon
./gradlew test
\end{lstlisting}
results are in: \url{build/reports/test/index.html}.

Running just one unit test can be accomplished by filtering using the \texttt{--test} flag to \texttt{gradlew} (see \url{https://docs.gradle.org/current/userguide/java_plugin.html#test_filtering}).


\subsubsection{Code coverage}
\index{Testing!code coverage}

% TODO add text about Jacoco
\index{Jacoco}



\subsubsection{Monitoring code quality}
\index{Testing!Monitoring code quality}

% TODO add text about SonarQube
\index{SonarQube}




\subsection{Integration testing}
\label{sec:integration-testing}
\index{Testing!integration testing}
\index{Xenon!sourceSet!integrationTest@\texttt{integrationTest}}
% TODO Integration testing section
Running integration tests (first time is slow, because it's downloading all the docker images)
\begin{lstlisting}[style=basic,style=bash,escapeinside={(*@}{@*)}]
cd (*@\mytilde@*)/Xenon
./gradlew dockerIntegrationTest
\end{lstlisting}
results are in: \url{build/reports/integrationTest/index.html}.


\url{https://www.codacy.com/app/NLeSC/Xenon}
\url{https://travis-ci.org/NLeSC/Xenon}


\subsection{Continuous integration testing}
\index{Testing!continuous integration testing}
