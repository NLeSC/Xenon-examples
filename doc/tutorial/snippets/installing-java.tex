\subsection{Installing Java}
\index{Installing Java}

Xenon is a Java library, therefore it needs Java in order to run. Java comes in different versions identified by a name and a number. The labeling is somewhat confusing\footnote{See for example \url{http://stackoverflow.com/questions/2411288/java-versioning-and-terminology-1-6-vs-6-0-openjdk-vs-sun}}. This is partly because Java was first developed by Sun Microsystems (which was later bought by Oracle Corporation), while an open-source implementation is also available (it comes standard with many Linuxes). Furthermore, there are different flavors for each version, each flavor having different capabilities. For example, if you just want to \textit{run} Java applications, you need the JRE\index{JRE} (Java Runtime Environment\index{Java Runtime Environment}); if you also want to \textit{develop} Java software, you'll need either an SDK\index{SDK} (Software Development Kit\index{Java Software Development Kit}) from Sun/Oracle, or a JDK\index{JDK} (Java Development Kit\index{Java Development Kit}) if you are using the open-source variant.

Check if you have Java and if so, what version you have:
\begin{lstlisting}[style=basic,style=bash]
java -version
\end{lstlisting}
That should produce something like:
\begin{lstlisting}[style=basic,style=bash]
java version "1.7.0_79"
OpenJDK Runtime Environment (IcedTea 2.5.6) (7u79-2.5.6-0ubuntu1.14.04.1)
OpenJDK 64-Bit Server VM (build 24.79-b02, mixed mode)
\end{lstlisting}
Note that `Java version 1.7' is often referred to as `Java 7'.

If you don't have Java yet, install it with:
\begin{lstlisting}[style=basic,style=bash]
sudo apt-get install default-jdk
\end{lstlisting}
