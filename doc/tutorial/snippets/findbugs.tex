\vspace{2em}
\textit{Static code analysis with FindBugs}
\index{Eclipse!FindBugs plugin}
\index{static code analysis}
\index{FindBugs}

The FindBugs plugin for Eclipse lets you analyze your Java code for bugs. FindBugs is particularly useful when you are relatively new to Java, because it provides feedback on what goes wrong in your code (a `bug pattern' in FindBugs terminology), and makes suggestions about how to change it.

In Eclipse's menu go to \textsf{Help} and select \textsf{Install New Software...} once more. Click \textsf{Add...}. For \textsf{Name}, type `FindBugs'; for \textsf{Location}, type \url{ http://findbugs.cs.umd.edu/eclipse}, then click \textsf{OK}. Make sure \textsf{FindBugs} is checked in the list, before clicking \textsf{Finish}. Restart Eclipse when prompted.

After the restart, you should be able to perform static code analysis. To do so, right-click the name of your project in the \textsf{Package Explorer} pane, there should be an item \textsf{Find Bugs}. Click it to start analyzing your code. After the analysis is done, you can view the results as follows: Go to \textsf{Window}$\rightarrow$\textsf{Show View}$\rightarrow$\textsf{Other...}. Look for the item labeled \textsf{FindBugs} and expand it. Select \textsf{Bug Explorer} and \textsf{Bug Info}. This should give you two extra panes with information of what type of bugs are in your code, where each bug is located, what makes it a bug, and what could be done to resolve it.

