
\subsection{Verifying the software setup}

To check if everything works, we first need to build the example from source and then run the example from the command line.

\subsubsection{Building}

The code is divided into a few sets of files that conceptually belong together. Such sets are referred to as `sourceSets'\index{sourceSets}. Each sourceSet has its own subdirectory; for example, the sourceSet `main'\index{Xenon!sourceSet!main@\texttt{main}} contains the source code for Xenon and is located at \url{src/main/java}. We'll also need a second sourceSet, `examples'\index{Xenon!sourceSet!examples@\texttt{examples}}, which contains the source code for the Xenon examples. It is located at \url{src/examples/java}. We will compile both sourceSets using a build automation tool, which comes in the form of a script called \texttt{gradlew}.

How a project should be built is defined in `*.gradle' files. Xenon has two main ones (`build.gradle' and `build-offline.gradle'), as well as a few auxiliary ones (located in the \texttt{gradle} subdirectory). Together, they define all kinds of neat things you can do with the source code, such as compiling, generating documentation, uploading generated jar files to an external repository, and so on. You can get an overview of what tasks are available by:

\begin{lstlisting}[style=basic,style=bash,escapeinside={(*@}{@*)}]
cd ${HOME}/Xenon
./gradlew tasks --all
\end{lstlisting} % dummy $

Under `Build tasks', there should be an item \texttt{examplesClasses}, used for compiling the `examples' sourceSet; \texttt{examplesClasses} has a dependent task \texttt{classes}, used for compiling the `main' sourceSet.

Running
\begin{lstlisting}[style=basic,style=bash,escapeinside={(*@}{@*)}]
cd ${HOME}/Xenon
./gradlew examplesClasses
\end{lstlisting} % dummy $
should give you a new directory \mytilde\url{/Xenon/build} with subdirectories \url{classes/examples} and {\url{classes/main} (as well as some other stuff).


\subsubsection{Running an example}

So at this point we have compiled the necessary Java classes; now we need to figure out how to run them.

The general syntax for running compiled Java programs from the command line\index{Java!from the command line} is as follows:
\begin{lstlisting}[style=basic,style=bash,escapeinside={(*@}{@*)}]
java (*@\textit{<fully qualified classname>}@*)
\end{lstlisting}
The fully qualified classname for our example is \url{nl.esciencecenter.xenon.examples.files.DirectoryListing}, but if you try to run
\begin{lstlisting}[style=basic,style=bash,escapeinside={(*@}{@*)}]
cd ${HOME}/Xenon
java nl.esciencecenter.xenon.examples.files.DirectoryListing
\end{lstlisting} % dummy $
you will get the error below:
\begin{lstlisting}[style=basic,style=bash,escapeinside={(*@}{@*)}]
Error: Could not find or load main class \
nl.esciencecenter.xenon.examples.files.DirectoryListing
\end{lstlisting}

This is because the \url{java} executable tries to locate our class \url{nl.esciencecenter.xenon.examples.files.DirectoryListing}, but we haven't told \url{java} where to look for it. We can resolve that by specifying a list of one or more directories using \texttt{java}'s classpath option \texttt{-cp}\index{Java!from the command line!classpath}\index{Java!from the command line!-cp@\texttt{-cp}}. There are 3 locations that are relevant for running \url{DirectoryListing}. These are:
\begin{enumerate}
\item{the location of \url{DirectoryListing} itself:\\ \mytilde\url{/Xenon/build/classes/examples}}
\item{the location of the Xenon classes:\\ \mytilde\url{/Xenon/build/classes/main}}
\item{the location of any libraries that Xenon depends on:\\ \mytilde\url{/Xenon/lib}}
\end{enumerate}
These directories can be passed to \texttt{java} as a colon-separated list, in which directory names can be relative to the current directory. Furthermore, the syntax is slightly different depending on what type of file you want \texttt{java} to find in a given directory: if you want \texttt{java} to find compiled Java classes, use the directory name; if you want \texttt{java} to find jar files, use the directory name followed by \texttt{/*}. Finally, the order within the classpath is significant.
% TODO find out about the order in java classpath

Using paths relative to \mytilde\url{/Xenon}, our classpath thus becomes \url{build/classes/examples:build/classes/main:lib/*}. However, if we now try to run
\begin{lstlisting}[style=basic,style=bash,escapeinside={(*@}{@*)}]
cd ${HOME}/Xenon
java -cp 'build/classes/examples:build/classes/main:lib/*' \
nl.esciencecenter.xenon.examples.files.DirectoryListing
\end{lstlisting}
% dummy $
it still does not work yet, because \url{DirectoryListing} takes exactly one input argument that defines the location (URI\index{URI}) of the directory whose contents we want to list. URIs generally consist of a \textit{scheme}\index{URI!scheme} followed by an \textit{authority}\index{URI!authority}. For a local file, the scheme is \texttt{file://}. The authority is the name of the directory we want to list the contents of, such as \texttt{\$PWD} (the present working directory).

Putting all that together, we get:

\begin{lstlisting}[style=basic,style=bash,escapeinside={(*@}{@*)}]
cd ${HOME}/Xenon
java -cp 'build/classes/examples:build/classes/main:lib/*' \
nl.esciencecenter.xenon.examples.files.DirectoryListing file://$PWD
\end{lstlisting} % dummy $

If all goes well, you should now see the contents of the current directory.


