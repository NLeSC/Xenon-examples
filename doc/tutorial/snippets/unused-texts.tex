\chapter{Unused texts}





\section{Docker}
\label{sec:docker}
\index{Docker}
\index{Docker!installing}

Later on in this document, we will take a closer look at continuous integration testing\index{continuous integration testing}. Usually, setting up a testing environment\index{testing!environment} is reasonably easy, but for Xenon it's a little more complicated. This is because the Xenon library is about connecting to remote systems, but in a testing environment, such remote systems do not exist---there's only the test machine. So, in order to run Xenon tests, we need to set up an environment in which a \textit{virtual} remote system\index{testing!virtual remote system} is used. Multiple virtual remote systems may in actual fact run on one \textit{physical} machine. For now, let's just install the Docker\footnote{The remainder of this section is based on: \url{https://docs.docker.com/engine/installation/ubuntulinux/}} software that we'll use to set up virtual remote systems. Note that Docker needs a 64-bit host system. Also, it needs a minimum kernel version of 3.10 (again, on the host).

\Needspace{4\baselineskip}
Check your kernel version with:
\begin{lstlisting}[style=basic,style=bash,escapeinside={(*@}{@*)}]
uname -r
\end{lstlisting}
Mine is \texttt{3.13.0-67-generic}.

The Ubuntu repositories contain an older version of Docker, which you should not use. Instead, use the newer version from Docker's own PPA\index{Docker!PPA}.

\Needspace{4\baselineskip}
First check if you have the older version by:
\begin{lstlisting}[style=basic,style=bash,escapeinside={(*@}{@*)}]
docker -v
\end{lstlisting}
Mine says:
\begin{lstlisting}[style=basic,style=bash,escapeinside={(*@}{@*)}]
Docker version 1.9.0, build 76d6bc9
\end{lstlisting}

If your version is lower, go ahead and uninstall as follows. First find out where your Docker program lives with:
\begin{lstlisting}[style=basic,style=bash,escapeinside={(*@}{@*)}]
which docker
\end{lstlisting}

\Needspace{4\baselineskip}
and then find out which package your Docker is a part of with:
\begin{lstlisting}[style=basic,style=bash,escapeinside={(*@}{@*)}]
dpkg -S `which docker`
\end{lstlisting}

If you already had Docker installed, then the package name is likely either \texttt{docker.io}\index{Docker!docker.io@\texttt{docker.io}} or \texttt{lxc-docker}\index{Docker!lxc-docker@\texttt{lxc-docker}}. Either way uninstall the entire package, including its settings with
\begin{lstlisting}[style=basic,style=bash,escapeinside={(*@}{@*)}]
sudo apt-get remove --purge docker.io
\end{lstlisting}
or
\begin{lstlisting}[style=basic,style=bash,escapeinside={(*@}{@*)}]
sudo apt-get remove --purge lxc-docker*
\end{lstlisting}

We will use software from a third-party repository, \url{https://apt.dockerproject.org}\index{Docker!PPA}. For this, we'll need to add the new repository's PGP key\index{Docker!PGP key} to our installation as follows:
\begin{lstlisting}[style=basic,style=bash,escapeinside={(*@}{@*)}]
sudo apt-key adv --keyserver hkp://pgp.mit.edu:80 \
--recv-keys 58118E89F3A912897C070ADBF76221572C52609D
\end{lstlisting}

The details of the next step vary depending on the operating system you are using, so let's first check which version you are running:
\begin{lstlisting}[style=basic,style=bash,escapeinside={(*@}{@*)}]
lsb_release -dc
\end{lstlisting}
Make a note of your distribution's codename for the next step (mine is \texttt{trusty}).

Open or create the file \url{/etc/apt/sources.list.d/docker.list} in an editor such as nano, gedit, leafpad, etc. I'm using nano:
\begin{lstlisting}[style=basic,style=bash,escapeinside={(*@}{@*)}]
sudo nano /etc/apt/sources.list.d/docker.list
\end{lstlisting}

\Needspace{6\baselineskip}
Delete any existing entries in \url{/etc/apt/sources.list.d/docker.list}, then add one of the following options
\begin{lstlisting}[style=basic,style=bash,escapeinside={(*@}{@*)}]
deb https://apt.dockerproject.org/repo ubuntu-precise main
deb https://apt.dockerproject.org/repo ubuntu-trusty main
deb https://apt.dockerproject.org/repo ubuntu-vivid main
deb https://apt.dockerproject.org/repo ubuntu-wily main
\end{lstlisting}
(I chose the second because I'm on trusty).

Next, save and close \url{/etc/apt/sources.list.d/docker.list}.

Now that we have added Docker's PPA\index{Docker!PPA} to the list of software sources, we need to update the list with the package information as follows:
\begin{lstlisting}[style=basic,style=bash,escapeinside={(*@}{@*)}]
sudo apt-get update
\end{lstlisting}

Check if your are now using the right docker\index{Docker!docker-engine@\texttt{docker-engine}}:
\begin{lstlisting}[style=basic,style=bash,escapeinside={(*@}{@*)}]
apt-cache policy docker-engine
\end{lstlisting}
Mine says:
\begin{lstlisting}[style=basic,style=bash,escapeinside={(*@}{@*)}]
docker-engine:
Installed: 1.9.0-0~trusty
Candidate: 1.9.0-0~trusty
Version table:
***1.9.0-0~trusty 0
     500 https://apt.dockerproject.org/repo/ ubuntu-trusty/main amd64 Packages
     100 /var/lib/dpkg/status
   1.8.3-0~trusty 0
     500 https://apt.dockerproject.org/repo/ ubuntu-trusty/main amd64 Packages
   1.8.2-0~trusty 0
     500 https://apt.dockerproject.org/repo/ ubuntu-trusty/main amd64 Packages
   1.8.1-0~trusty 0
     500 https://apt.dockerproject.org/repo/ ubuntu-trusty/main amd64 Packages
   1.8.0-0~trusty 0
     500 https://apt.dockerproject.org/repo/ ubuntu-trusty/main amd64 Packages
   1.7.1-0~trusty 0
     500 https://apt.dockerproject.org/repo/ ubuntu-trusty/main amd64 Packages
   1.7.0-0~trusty 0
     500 https://apt.dockerproject.org/repo/ ubuntu-trusty/main amd64 Packages
   1.6.2-0~trusty 0
     500 https://apt.dockerproject.org/repo/ ubuntu-trusty/main amd64 Packages
   1.6.1-0~trusty 0
     500 https://apt.dockerproject.org/repo/ ubuntu-trusty/main amd64 Packages
   1.6.0-0~trusty 0
     500 https://apt.dockerproject.org/repo/ ubuntu-trusty/main amd64 Packages
   1.5.0-0~trusty 0
     500 https://apt.dockerproject.org/repo/ ubuntu-trusty/main amd64 Packages
\end{lstlisting}

Now for the actual install. If your Ubuntu version is Ubuntu Wily 15.10, Ubuntu Vivid 15.04, or Ubuntu Trusty 14.04 (LTS), you're in luck, as these OS'es have everything you'll need already. If you're not on one of these Ubuntu versions, refer to \url{https://docs.docker.com/engine/installation/ubuntulinux/} for instructions on installing some additional packages before proceeding with the next step.

Install Docker with\index{Docker!installing}:
\begin{lstlisting}[style=basic,style=bash,escapeinside={(*@}{@*)}]
sudo apt-get install docker-engine
\end{lstlisting}

\Needspace{4\baselineskip}
The Docker service should have started; if for some reason it hasn't, you can start it manually by\index{Docker!start service}:
\begin{lstlisting}[style=basic,style=bash,escapeinside={(*@}{@*)}]
sudo service docker start
\end{lstlisting}

Now let's try a small example to see if Docker works\index{Docker!hello world}:
\begin{lstlisting}[style=basic,style=bash,escapeinside={(*@}{@*)}]
sudo docker run hello-world
\end{lstlisting}

This command downloads a test image \texttt{hello-world} from DockerHub\index{Docker!DockerHub}, an external repository for storing Docker images\index{Docker!image}. Just to be clear, an `image' in this context refers to an image of an operating system---it has nothing to do with a picture.

% TODO add what is a container
When the container runs, it prints an informational message. Then, it exits.

You can check where docker images are stored by\index{Docker!docker info@\texttt{docker info}}:
\begin{lstlisting}[style=basic,style=bash,escapeinside={(*@}{@*)}]
docker info
\end{lstlisting}
Mine are stored under \url{/var/lib/docker}; whatever the location, make sure you have enough disk space there, as Docker will download any new containers to that location.

The Docker daemon binds to a Unix socket instead of a TCP port. By default that Unix socket is owned by the user \texttt{root} and other users can access it with \texttt{sudo}. For this reason, the Docker daemon always runs as the \texttt{root} user.

To avoid having to use \texttt{sudo} when you use the \texttt{docker} command, we will create a Unix group called \texttt{docker} and add users to it. When the Docker daemon starts, it makes the ownership of the Unix socket read/writable by the \texttt{docker} group.

Add yourself to the \texttt{docker} group\index{Docker!the docker group@the \texttt{docker} group} with:
\begin{lstlisting}[style=basic,style=bash,escapeinside={(*@}{@*)}]
sudo usermod -G docker -a <name-of-user>
\end{lstlisting}
Log out and back in.

We will use multiple Docker containers simultanenously. To coordinate how individual Docker containers talk to each other, we need a tool called \texttt{docker-compose}\footnote{For more information on installation, see: \url{https://docs.docker.com/compose/install/}}\index{Docker!docker-compose@\texttt{docker-compose}}. It uses
a so-called compose file\index{Docker!compose file} to configure an container's services. Xenon's compose file is \url{docker-compose.yml} located in \url{src/integrationTest/docker/}.

To install the \texttt{docker-compose} program, first check \url{https://github.com/docker/compose/releases} to see what the latest stable version of \texttt{docker-compose} is. This determines the \texttt{VERSION\_NUM} in the command below. Mine is \texttt{1.5.0}.

\Needspace{6\baselineskip}
Download \texttt{docker-compose}\index{Docker!docker-compose@\texttt{docker-compose}} using \texttt{curl} form the terminal:
\begin{lstlisting}[style=basic,style=bash,escapeinside={(*@}{@*)}]
cd (*@\mytilde@*)
curl -L https://github.com/docker/compose/releases/downlo\
ad/VERSION_NUM/docker-compose-`uname -s`-`uname -m` > docker-compose
\end{lstlisting}

Then move the downloaded file into the right directory on your system with:
\begin{lstlisting}[style=basic,style=bash,escapeinside={(*@}{@*)}]
cd (*@\mytilde@*)
sudo mv docker-compose /usr/local/bin/
\end{lstlisting}

Apply executable permissions to the binary:
\begin{lstlisting}[style=basic,style=bash,escapeinside={(*@}{@*)}]
sudo chmod +x /usr/local/bin/docker-compose
\end{lstlisting}

And verify that it worked:
\begin{lstlisting}[style=basic,style=bash,escapeinside={(*@}{@*)}]
docker-compose --version
\end{lstlisting}
Mine says:
\begin{lstlisting}[style=basic,style=bash,escapeinside={(*@}{@*)}]
docker-compose version: 1.5.0
\end{lstlisting}

That's it for now, we will get back to Docker in Section~\ref{sec:testing}:~\nameref{sec:testing}.














